\documentclass{article}
\usepackage[utf8x]{inputenc}
\usepackage[T1]{fontenc}
\usepackage[MeX]{polski}

\title{Specyfikacja funkcjonalna automatu kom\'orkowego \texttt{GameOfLife}:
	Gra w \.zycie Johna Conwaya}
\author{Aleksandra Michalska, Natalia Olszweska}
\date{09.03.2021}

\begin{document}

\begin{titlepage}
	\maketitle
\end{titlepage}

\section{Opis og\'olny}
\subsection{Nazwa programu}
\quad Nazwa programu to \texttt{\textit{"GameOfLife"}}.
\subsection{Wst\k{e}p teoretyczny}
\quad Gra w \.zycie Johna Conwaya jest automatem kom\'orkowym, czyli systemem sk\l{}adaj\k{a}cym si\k{e} z pojedynczych kom\'orek. Ka\.zda taka kom\'orka znajduje si\k{e} w jednym ze sko\'nczonej liczby stan\'ow (mo\.ze by\'c martwa lub \.zywa). 
\subsection{Cel projektu}
\quad Program ma na celu wy\'swietlanie kolejnych generacji gry w \.zycie przy u\.zyciu konsoli systemowej. Program mo\.ze dzia\l{}a\'c zar\'owno w trybie interaktywnym jak i wsadowym. Wybrane obrazy generowane przez program zapisywane mog\k{a} by\'c do pliku o rozszerzeniu graficznym.
\subsection{Cel dokumentu}
\quad Dokument ma na celu przybli\.zenie ko\.zystania z programu jego u\.zytkownikowi docelowemu.
\subsubsection{U\.zytkownik docelowy}
\quad Program jest powszechnie dost\k{e}pny oraz dedykowany jest dla ka\.zdego u\.zytkownika.


\section{Opis funkcjonalno\'sci}
\subsection{Jak ko\.zysta\'c z programu?}
\subsection{Argumenty wywo\l{}ania programu}
\subsection{Mo\.zliwo\'sci programu}

\section{Format danych i struktura plik\'ow}
\subsection{Poj\k{e}cia i pola formularza}
\subsection{Struktura katalog\'ow}
\subsection{Przechowywanie danych w programie}
\subsection{Dane wej\'sciowe}
Do poprawnego dzia\l{}ania programu potrzebne jest podanie na wej\'sciu nast\k{e}puj\k{a}cych parametr\'ow oraz danych:
\begin{itemize}
	\item \texttt{--data filename} nazwa pliku z danymi wej\'sciowymi. W pilku powinny znajdowa\'c si\k{e} nast\k{e}puj\k{a}ce dane: 
		\begin{itemize}
			\item ilo\'s\'c wierszy (liczba ca\l{}kowita z zakresu od 3 do 30)
			\item ilo\'s\'c kolumn (liczba ca\l{}kowita z zakresu od 3 do 30)
		\end{itemize}
\end{itemize}

\subsection{Dane wyj\'sciowe}

\section{Scenariusz dzia\l{}ania programu}
\subsection{Scenariusz og\'olny}
\subsection{Scenariusz szczeg\'o\l{}owy}
\subsection{Ekrany dzia\l{}ania programu????????????????}
\subsection{Komunikaty b\l{}ed\'ow}


\end{document}
