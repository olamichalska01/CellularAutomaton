\documentclass{article}
\usepackage[utf8x]{inputenc}
\usepackage[T1]{fontenc}
\usepackage[MeX]{polski}
\usepackage{amssymb}
\usepackage{graphicx}
\usepackage{fancyhdr}
\usepackage{lastpage}

\pagestyle{fancy}
\fancyhf{}
\cfoot{ \thepage \hspace{1pt} / \pageref{LastPage}}
\lhead{Specyfikacja funkcjonalna}
\rhead{\texttt{GameOfLife}}


\begin{document}

\begin{titlepage}
	\begin{center}
		\vspace*{5cm}

	        \Huge
        	\textbf{Specyfikacja funkcjonalna automatu kom\'orkowego}

        	\vspace{1cm}
	        \Huge
        	\texttt{GameOfLife}: Gra w \.zycie Johna Conwaya

    		\vspace{1.5cm}

	        \large
		Aleksandra Michalska, Natalia Olszweska

        	\vfill

	        \vspace{3cm}


		\large 09.03.2021
	\end{center}
\end{titlepage}



\section{Opis og\'olny}
\subsection{Nazwa programu}
\quad Nazwa programu to \texttt{\textit{"GameOfLife"}}.
\subsection{Wst\k{e}p teoretyczny}
\quad Gra w \.zycie Johna Conwaya jest automatem kom\'orkowym, czyli systemem sk\l{}adaj\k{a}cym si\k{e} z pojedynczych kom\'orek. Ka\.zda taka kom\'orka znajduje si\k{e} w jednym ze sko\'nczonej liczby stan\'ow (mo\.ze by\'c martwa lub \.zywa). 
\subsection{Cel projektu}
\quad Program ma na celu wy\'swietlanie kolejnych generacji gry w \.zycie przy u\.zyciu konsoli systemowej. Program mo\.ze dzia\l{}a\'c zar\'owno w trybie interaktywnym jak i wsadowym. Wybrane obrazy generowane przez program zapisywane mog\k{a} by\'c do pliku o rozszerzeniu graficznym.
\subsection{Cel dokumentu}
\quad Dokument ma na celu przybli\.zenie ko\.zystania z programu jego u\.zytkownikowi docelowemu.
\subsubsection{U\.zytkownik docelowy}
\quad Program jest powszechnie dost\k{e}pny oraz dedykowany jest dla ka\.zdego u\.zytkownika.




\section{Opis funkcjonalno\'sci}
\subsection{Jak ko\.zysta\'c z programu?}
\subsection{Argumenty wywo\l{}ania programu}
Do poprawnego dzia\l{}ania programu potrzebne jest podanie na wej\'sciu nast\k{e}puj\k{a}cych parametr\'ow oraz danych:
\begin{itemize}
	\item \textbf{\texttt{ -in filein.txt}} nazwa pliku z danymi wej\'sciowymi
	\item \textbf{\texttt{ -out fileout.txt}} nazwa pliku do kt\'orego zapisywana b\k{e}dzie ko\'ncowa generacja programu
	\item \textbf{\texttt{ -s(o5 || f5) }}:
		\begin{itemize}
			\item \textbf{\texttt{ -s(o5)}} "save one" - zapisuje jedn\k{a} \texttt{n}- t\k{a} generacj\k{e} obrazu do pilku 
			\item \textbf{\texttt{ -s(f5)}} "save first" - zapisuje pierwsze \texttt{n} obraz\'ow do plik\'ow
		\end{itemize}
	\item \textbf{\texttt{ -m(sbs || fast)}}:
		\begin{itemize}
			\item \textbf{\texttt{ -m(sbs)}} "step-by-step mode" - tryb krok po kroku; u\.zytkownik naicskaj\k{a}c dowolny klawisz przechodzi do kolejnej generacji. Istnieje mo\.zliwo\'s\'c przej\'scia z trybu \texttt{\textbf{sbs}} do trybu \texttt{\textbf{fast}} naciskaj\k{a}c klawisz \texttt{\textbf{e}}
			\item \textbf{\texttt{ -m(fast)}} "fast mode" - tryb szybki; kolejne generacje wy\'swietlaj\k{a} si\k{e} automatycznie.
		\end{itemize}

\end{itemize}
\subsection{Dane wej\'sciowe}
Dane wej\'sciowe s\k{a} przekazywane do programu w pliku tekstowym o rozszerzeniu .txt . W pilku powinny znajdowa\'c si\k{e} nast\k{e}puj\k{a}ce dane: 
\begin{itemize}
	\item \texttt{w} ilo\'s\'c wierszy $W = \{ w \in \mathbb{Z} : 3 \leq w \leq 30 \} $
	\item \texttt{k} ilo\'s\'c kolumn $K = \{ k \in \mathbb{Z} : 3 \leq k \leq 30 \} $
	\item wype\l{}nienie ka\.zdej kom\'orki: 0(kom\'orka martwa) lub 1(kom\'orka \.zywa)
		
		Przykladowe dane z pliku wejsciowego, wype\l{}niaj\k{a}ce tabel\k{e} o 4 wierszach i 4 kolumnach:
		\begin{table}[h!]
		\begin{center}
			\begin{tabular}{c c c c}
				4 & 4         \\
				0 & 0 & 1 & 0 \\
				0 & 1 & 0 & 1 \\
				1 & 0 & 0 & 0 \\
				0 & 1 & 1 & 0 \\
			\end{tabular}
		\end{center}
		\end{table}

		Generuj\k{a} poni\.zszy obraz pocz\k{a}tkowy:

		\begin{figure}[h]
			\centering
			\includegraphics{obraz}
		\end{figure}
\end{itemize}

\subsection{Dane wyj\'sciowe}


\section{Scenariusz dzia\l{}ania programu}
\subsection{Scenariusz og\'olny}
\subsection{Scenariusz szczeg\'o\l{}owy}
\subsection{Komunikaty b\l{}ed\'ow}


\end{document}
