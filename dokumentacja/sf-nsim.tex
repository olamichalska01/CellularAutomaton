\documentclass[]{article}
\usepackage{lmodern}
\usepackage{amssymb,amsmath}
\usepackage{ifxetex,ifluatex}
\usepackage{fixltx2e} % provides \textsubscript
\ifnum 0\ifxetex 1\fi\ifluatex 1\fi=0 % if pdftex
  \usepackage[T1]{fontenc}
  \usepackage[utf8]{inputenc}
\else % if luatex or xelatex
  \ifxetex
    \usepackage{mathspec}
  \else
    \usepackage{fontspec}
  \fi
  \defaultfontfeatures{Ligatures=TeX,Scale=MatchLowercase}
\fi
% use upquote if available, for straight quotes in verbatim environments
\IfFileExists{upquote.sty}{\usepackage{upquote}}{}
% use microtype if available
\IfFileExists{microtype.sty}{%
\usepackage[]{microtype}
\UseMicrotypeSet[protrusion]{basicmath} % disable protrusion for tt fonts
}{}
\PassOptionsToPackage{hyphens}{url} % url is loaded by hyperref
\usepackage[unicode=true]{hyperref}
\hypersetup{
            pdftitle={Specyfikacja funkcjonalna symulatora układu N-ciał: nsim},
            pdfauthor={Bartosz Chaber},
            pdfborder={0 0 0},
            breaklinks=true}
\urlstyle{same}  % don't use monospace font for urls
\IfFileExists{parskip.sty}{%
\usepackage{parskip}
}{% else
\setlength{\parindent}{0pt}
\setlength{\parskip}{6pt plus 2pt minus 1pt}
}
\setlength{\emergencystretch}{3em}  % prevent overfull lines
\providecommand{\tightlist}{%
  \setlength{\itemsep}{0pt}\setlength{\parskip}{0pt}}
\setcounter{secnumdepth}{0}
% Redefines (sub)paragraphs to behave more like sections
\ifx\paragraph\undefined\else
\let\oldparagraph\paragraph
\renewcommand{\paragraph}[1]{\oldparagraph{#1}\mbox{}}
\fi
\ifx\subparagraph\undefined\else
\let\oldsubparagraph\subparagraph
\renewcommand{\subparagraph}[1]{\oldsubparagraph{#1}\mbox{}}
\fi

% set default figure placement to htbp
\makeatletter
\def\fps@figure{htbp}
\makeatother


\title{Specyfikacja funkcjonalna symulatora układu N-ciał: \texttt{nsim}}
\author{Bartosz Chaber}
\date{27.02.2019}

\begin{document}
\maketitle

\section{Cel projektu}\label{header-n231}

Program \texttt{nsim} ma na celu numeryczne wyznaczanie trajektorii
obiektu, na który wpływa N-1 innych ciał. Program działa w trybie
nieinteraktywnym/wsadowym. Wygenerowane zbiory danych pozwalają na ich
wizualizację w programie \texttt{gnuplot}. Program umożliwia analizę
wielu ciał poprzez umożliwienie pominięcia nieistotnych oddziaływań. W
\texttt{nsim} zaimplementowano różne sposoby obsługi kolizji. Użytkownik
może wybrać też metodę całkowania numerycznego.

\section{Dane wejściowe}\label{header-n233}

Do swojego działania program potrzebuje danych o śledzonych ciałach.
Każde z ciał musi być opisane następującymi parametrami:

\begin{itemize}
\item
  etykieta/nazwa: nie jest używana w symulacji, jednak jest zapisywana
  do pliku wyjściowego w celach wizualizacyjnych; etykieta może zawierać
  znaki \texttt{a-zA-Z0-9\ -} oraz nie może być dłuższa niż 32 znaki;
\item
  masa ciała (M): podana w kg, dopuszczalne wartości M \textgreater{} 0;
\item
  położenie w przestrzeni 3D (x, y, z): niezbędne jest zdefiniowanie
  położenia \emph{x} i \emph{y}. W przypadku niezdefiniowanego położenia
  \emph{z}, program założy z=0; położenie podane w metrach, względem
  pewnego początku układu odniesienia;
\item
  bieżąca prędkość w przestrzeni 3D (vx, vy, vz): analogicznie jak w
  przypadku położenia, niezdefiniowana prędkość \emph{vz} będzie
  zinterpretowana jako vz=0; prędkości podane są w m/s;
\item
  promień obiektu (R): podany w metrach, dopuszczalne tylko R
  \textgreater{} 0.
\end{itemize}

Parametry symulacji numerycznej są modyfikowane za pomocą argumentów
wywołania programu.

Każde z ciał jest zdefiniowane w jednej linii, jego parametry oddzielone
są od siebie znakiem \texttt{;}. Program wczytuje kolejne ciała aż do
napotkania linii zawierającej tylko znak \texttt{.}.

Przykład danych wejściowych:

\begin{verbatim}
object_a;10; 1.5 2.0;15.0 0.0 0.0;0.2
object_b; 4;-0.5 0.0;-15.0 0.0;0.2
.
\end{verbatim}

W powyższym pliku zdefiniowane są dwa ciała \texttt{object\_a} i
\texttt{object\_b} o masach (odpowiednio) 10 kg i 4 kg. Położenie
obiektu \texttt{object\_a} to {[}1,5m; 2m; 0m{]}, \texttt{object\_b} to
{[}-0,5m; 0m; 0m{]}. Prędkości obiektów to:

\begin{itemize}
\item
  \texttt{object\_a} \textless{}15m/s, 0m/s, 0m/s\textgreater{},
\item
  \texttt{object\_a} \textless{}-15m/s, 0m/s, 0m/s\textgreater{}.
\end{itemize}

\section{Argumenty wywołania programu}\label{header-n256}

Program \texttt{nsim} akceptuje następujące argumenty wywołania:

\begin{itemize}
\item
  \texttt{-\/-timestep\ dt} precyzuje krok czasowy wykorzystywany
  podczas całkowania numerycznego; podany w sekundach; domyślnie
  \texttt{dt\ =\ 60};
\item
  \texttt{-\/-iterations\ n} określa liczbę analizowanych kroków
  czasowych; domyślnie \texttt{n\ =\ 1000}, liczba iteracji nie może być
  mniejsza od 1;
\item
  \texttt{-\/-integrator\ euler\textbar{}verlet} pozwala na wybór
  algorytmu całkowania numerycznego: \texttt{euler} określa metodę
  całkowania Eulera, \texttt{verlet} określa metodę całkowania Verleta;
  domyślna wartość to \texttt{euler};
\item
  \texttt{-\/-collision\ none\textbar{}remove\textbar{}momentum-exchange\textbar{}sticky}
  określa sposób obsługi kolizji między ciałami: \texttt{none} wyłącza
  kolizje, \texttt{remove} usuwa obydwa ciała,
  \texttt{momentum-exchange} odpowiada zderzeniu sprężystemu,
  \texttt{sticky} odpowiada zderzeniu niesprężystemu;
\item
  \texttt{-\/-relative-mass\ ratio} próg stosunku mas ciał poniżej
  którego pomija się wpływ ciał lżejszych; domyślna wartość
  \texttt{ratio=0.01};
\item
  \texttt{-\/-radius-of-influence\ R} odległość powyżej której nie
  uwzględniamy wpływu ciała; domyślna wartość \texttt{R=1e8};
\item
  \texttt{-\/-bodies\ filename} nazwa pliku, z którego wczytane będą
  parametry ciał (opisane w \emph{Dane wejściowe}); domyślnie jest to
  wartość pusta, oznaczająca pobieranie danych ze standardowego wejścia;
\item
  \texttt{-\/-output\ filename} nazwa pliku, do którego zapisane zostaną
  dane;
\end{itemize}

Przykładowe wywołania programu:

\begin{itemize}
\item
  \texttt{./nsim\ -\/-timestep\ 3600\ -\/-bodies\ solar.dat\ -\/-output\ solar.out}
  , efektem będzie analiza pliku \texttt{solar.dat} z krokiem czasowym 1
  godzina, dane wynikowe zostaną zapisane do pliku \texttt{solar.out}.
  Program nie będzie analizować kolizji. Zostanie zastosowana metoda
  całkowania Eulera. 
\end{itemize}

\section{Teoria}\label{header-n279}

{[}tutaj opisałbym jak to jest liczone -\/- patrz tablica{]}

\section{Komunikaty błędów}\label{header-n281}

Program \texttt{nsim} stara się kontynuować pracę mimo napotkania
nieprawidłowych danych w celu poprawienia diagnostyki danych
wejściowych.

\begin{enumerate}
\def\labelenumi{\arabic{enumi}.}
\item
  Masa ciała w danych wejściowych mniejsza/równa 0:
  \texttt{Linia\ 31:\ Masa\ ciała\ musi\ być\ większa\ od\ 0\ kg.\ Wczytano:\ "-2.1".\ Ciało\ zostało\ pominięte.}
  Program wykrył ujemną masę (o wartości \emph{-2.1}) w linii 31 danych
  wejściowych. Program ignoruje nieprawidłowe ciało.
\item
  Brak poprawnie wczytanych ciał:
  \texttt{W\ podanych\ danych\ wejściowych\ brak\ poprawnie\ zdefiniowanych\ ciał.\ Przerywam\ działanie.}

  Komunikat pojawia się, gdy w podanych danych wejściowych program nie
  znajdzie poprawnie zdefiniowanych ciał. Może to wynikać z tego, że
  plik jest pusty, albo wszystkie wpisy zawierając błędy (wtedy pojawią
  się stosowne komunikaty jak w 1.).
\end{enumerate}

\end{document}
