\documentclass{article}
\usepackage[utf8x]{inputenc}
\usepackage[T1]{fontenc}
\usepackage[MeX]{polski}

\title{Specyfikacja implementacyjna automatu kom\'orkowego \texttt{GameOfLife}:
	Gra w \.zycie Johna Conwaya}
\author{Aleksandra Michalska, Natalia Olszewska}
\date{09.03.2021}

\begin{document}
\begin{titlepage}
	\maketitle
\end{titlepage}

\section{Informacje og\'olne}

\subsection{Opis dokumentu}

\quad Dany dokument koresponduje z poprzedni\k{a} specyfikacj\k{a} - Secyfikacj\k{a} funkcjonaln\k{a}

\subsection{\'Srodowisko implementacyje}
\quad Docelowa gra zosta\l{}a napisana w \'srodowiskach Linuxowych, w j\k{e}zyku C. 


\subsection{Informacje o wy\'swietlanych generacjach}

\quad Wy\'swietlana posta\'c generacji zosta\l{}a utowrzona z dwukolorowych kwadra\'ow. 
Stan \.zywy kom\'orki zosta\l{} oznaczony bia\l{}ym kolorem, natomiast stan martwy kolorem czarnym. 
Kolorowe kwadraty zosta\l{}y wy\'swietlone przy pomocy Unicodu, \.zywa - u2b1c, martwa - u2b1b. 

\section{Opis modu\l{}\'ow - kr\'otki}

\subsection{Main}

\quad W module g\l{}\'ownym, dalej \textit{Main}, zosta\l{}a zaimplementowana podstawowa cz\k{e}\'s\'c programu. 
Mo\.za tam odczyta\'c posta\'c struktury pojedy\'nczej \textit{generacji}. 
\textit{Main} odpowiada za obs\l{}ug\k{e} parametr\'ow, czytanie pliku wej\'sciowego oraz zapis ko\'ncowej genracji, a tak\.ze uruchomienie odpowiedniego trybu programu (SBS, FAST). 
Modu\l{} g\l{}\'owny nawi\k{a}zuje do modu\l{}u \textit{Modes}.  

\subsection{Modes}

\quad W danym module zosta\l{}y zaimplementowne funkcje obs\l{}ugi poszczeg\'olnych tryb\'ow. 
Nawi\k{a}zuje on do modu\l{}\'ow \textit{SaveImage} oraz \textit{CreateNew}. 

\subsection{SaveImage}

\quad W tej cz\k{e}\'sci programu zosta\l{}y utworzone funkcje zapisu danej \textit{generacji} do postaci obrazu. 
Dany modu\l{} nie korzysta z \.zanych innych modu\l{}\'ow.

\subsection{CreateNew}

\quad W module znajduje si\k{e} funkcja tworz\k{a}ca now\k{a} \textit{generacje}, wed\l{}ug zasad opisanych r\'ownie\.z w danym module.
Nawi\k{a}zuje do modu\l{}u \textit{HelpCreate}.

\subsection{HelpCreate}

\quad W tym module zosta\l{}y zaimplementowane funkcje pomocnicze przy tworzeniu nowej \textit{generacji}.
Nawi\k{a}zuje do modu\l{}u \textit{Neighbor}.

\subsection{Neighbor}

\quad W tej cz\k{e}\'sci napisane zosta\l{}y funkcje potrzebne do sprawdzania liczby s\k{a}siad\'ow pojedy\'nczej kom\'orki.
Ten modu\l{} nie powo\l{}uje si\k{e} na inne modu\l{}y.

\section{Opis modu\l{}\'ow - szczeg\'o\l{}owy}


\section{Testowanie}


\section{Diagram modu\l{}\'ow}


\end{document}
